\documentclass[12pt]{book}

\usepackage{ae,lmodern}
\usepackage[utf8]{inputenc}
\usepackage[T1]{fontenc}
\usepackage[french]{babel}

\usepackage{amsmath}
\usepackage{amsfonts}
\usepackage{amssymb}
\usepackage{mathtools}
\renewcommand{\d}{\mathrm{d}}
\renewcommand{\j}{\mathrm{j}}
\newcommand{\e}{\mathrm{e}}
\newcommand{\ubar}[1]{\text{\b{$#1$}}}

\usepackage{amsthm}
\newtheorem{lemma}{Lemme}[chapter]
\theoremstyle{definition}
\newtheorem{definition}{Définition}[chapter]
\newtheorem{corollary}{Corollaire}[chapter]
\theoremstyle{remark}
\newtheorem{example}{Exemple}[chapter]

\usepackage{graphicx}
\usepackage{booktabs}

\usepackage{enumitem}

\usepackage{import}

%% Tikz
\usepackage{pgfplots}
\pgfplotsset{compat=1.16}
\usetikzlibrary{external}
\tikzexternalize
\tikzsetexternalprefix{tikz_figs/}
\tikzset{ressort/.style={thick,gray,smooth}}

\usepackage[french]{minitoc}

\begin{document}
\begin{titlepage}
    \centering
    \includegraphics[width=0.15\textwidth]{img/logo_lycee_michel_montaigne.png}\par\vspace{1cm}
    {\scshape Lycée Michel Montaigne \par}
    \vspace{1cm}
    {\scshape\Large Notes de cours\par}
    \vspace{1.5cm}
    {\huge\bfseries Physique-Chimie L2\par}
    \vspace{2cm}
    {\Large\itshape Régis Santet\par}
    \vfill
    Cours réalisé par\par
    Professeur~N. \textsc{Choimet}

    \vfill

    % Bas de page
    {\large Année scolaire 2015/2016\par}
\end{titlepage}

\dominitoc % Initialisation
\tableofcontents

% Première partie : mécanique
\subimport{cours/}{mecanique.tex}

    

\end{document}