\chapter{Référentiels non galiléens}

La description du mouvement d'un objet dépend de l'observateur. Un observateur lié à un solide (par exemple un train ou un quai) est lié à un système d'axes, c'est-à-dire 3 axes rigidement liés ainsi que d'une horloge (unique en mécanique classique car le temps est universel).

En L1, on étudie les référentiels galiléens vérifiant le principe d'inertie (1ère loi de Newton). En L2, on étudie les référentiels non galiléens dans deux cas :
\begin{itemize}
    \item les référentiels en translation accélérée,
    \item les référentiels en rotation uniforme autour d'un axe fixe.
\end{itemize}

\minitoc 

% Première section : Description du mouvement d'un point matériel
\subimport{referentiels_non_galileens/}{1_description_mouvement_point_materiel.tex}

% Deuxième section : Lois de la dynamique du point
\subimport{referentiels_non_galileens/}{2_lois_dynamique_point.tex}