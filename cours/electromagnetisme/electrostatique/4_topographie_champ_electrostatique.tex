\section{Topographie du champ électrostatique}

    \subsection{Lignes de champ. Tubes de champ}

        \paragraph{Ligne de champ.}
        Une ligne de champ est une courbe tangente en tout point au champ électrostatique $\vec{E}$. Quelques exemples sont donnés à la Figure~\ref{fig:ligne_de_champ_electrostatique}.

        \begin{figure}
            \centering
            \tikzsetnextfilename{ligne_de_champ_electrostatique}
            \begin{tikzpicture}[scale=0.75]  
                \coordinate (q) at (-5,5);
                \node at (q) {$\bullet$};
                \node at (q) [shift={(0,3)}] {$q>0$};
                \draw[dashed,draw=green] (q) circle (0.25cm);
                \draw[dashed,draw=green] (q) circle (0.75cm);
                \draw[dashed,draw=green] (q) circle (1.25cm);
                \draw[draw=blue] (q)--++(0:2);
                \draw[draw=blue] (q)--++(45:2);
                \draw[draw=blue] (q)--++(90:2);
                \draw[draw=blue] (q)--++(135:2);
                \draw[draw=blue] (q)--++(180:2);
                \draw[draw=blue] (q)--++(-135:2);
                \draw[draw=blue] (q)--++(-90:2);
                \draw[draw=blue] (q)--++(-45:2);
                \draw[draw=red, -latex] (q)--++(0:1);
                \draw[draw=red, -latex] (q)--++(45:1);
                \draw[draw=red, -latex] (q)--++(90:1);
                \draw[draw=red, -latex] (q)--++(135:1);
                \draw[draw=red, -latex] (q)--++(180:1);
                \draw[draw=red, -latex] (q)--++(-135:1);
                \draw[draw=red, -latex] (q)--++(-90:1);
                \draw[draw=red, -latex] (q)--++(-45:1);

                \draw (2.5,5)--++(5,0) node [right] {$\sigma>0$};
                \draw[draw=blue] (3,3)--++(0,4);
                \draw[draw=blue] (5,3)--++(0,4);
                \draw[draw=blue] (7,3)--++(0,4);
                \draw[draw=red, latex-latex] (3,4)--++(0,2);
                \draw[draw=red, latex-latex] (5,4)--++(0,2);
                \draw[draw=red, latex-latex] (7,4)--++(0,2);
                \draw[draw=green, dashed] (2.75,3.5)--++(4.5,0);
                \draw[draw=green, dashed] (2.75,4.5)--++(4.5,0);
                \draw[draw=green, dashed] (2.75,5.5)--++(4.5,0);
                \draw[draw=green, dashed] (2.75,6.5)--++(4.5,0);

                \draw (-8,-3)--(-7,-2)--(-3,-2)--(-4,-3)--cycle node [left]{$\sigma>0$};
                \draw (-8,-7)--(-7,-6)--(-3,-6)--(-4,-7)--cycle node [left]{$-\sigma<0$};
                \draw[dashed, draw=green] (-7.5,-5.5)--++(4,0);
                \draw[dashed, draw=green] (-7.5,-4.5)--++(4,0);
                \draw[dashed, draw=green] (-7.5,-3.5)--++(4,0);
                \draw[draw=blue] (-7,-3)--++(0,-3.5);
                \draw[draw=blue] (-5.5,-3)--++(0,-3.5);
                \draw[draw=blue] (-4,-3)--++(0,-3.5);
                \draw[draw=red,-latex] (-7,-3)--++(0,-2);
                \draw[draw=red,-latex] (-5.5,-3)--++(0,-2);
                \draw[draw=red,-latex] (-4,-3)--++(0,-2);
  
                \coordinate (q1) at (3,-4);
                \coordinate (q2) at (7,-6);
                \node at (q1) {$\bullet$};
                \node at (q2) {$\bullet$};
                \node at (q1) [shift={(0,1)}] {$-q$};
                \node at (q2) [shift={(0,1)}] {$+q>0$};
                \draw[dashed, draw=green] (q1) circle (0.75);
                \draw[dashed, draw=green] (q2) circle (0.75);
                \draw[blue, ->-=red] (q2) to [bend right=20] (q1);
                \draw[blue, ->-=red] (q2) to [bend left=20] (q1);
                \draw[blue, ->-=red] (q2) to [bend left=40] (q1);
                \draw[blue, ->-=red] (q2) to [bend right=40] (q1);
                \draw[blue, ->-=red] (q2) to [bend left=70] (q1);
                \draw[blue, ->-=red] (q2) -- (q1);
                \centerarc[dashed,draw=green](q1)(50:-120:1.5)
                \centerarc[dashed,draw=green](q2)(90:250:1.5)                                  
            \end{tikzpicture}
            \caption[Topographie du champ électrostatique : exemples de lignes de champ.]{Topographie du champ électrostatique : exemples de lignes de champ. Les courbes en vert sont les équipotentielles.}
            \label{fig:ligne_de_champ_electrostatique}
        \end{figure}

        Les lignes de champ divergent à partir des $q>0$ et convergent vers les $q<0$.

        \paragraph{Tube de champ.}
        Un tube de champ est un ensemble de lignes de champ s'appuyant sur un contour fermé.

    \subsection{Surfaces équipotentielles}

        Une surface équipotentielle est définie par
        \begin{equation*}
            \left\lbrace M\,\middle|\, V(M)=\text{constante}\right\rbrace.
        \end{equation*}
        Sur une surface équipotentielle, on a $\rmd V=0=-\vec{E}\cdot\vec{\rmd l}$. Ceci vaut pour tout déplacement le long de la surface $\vec{\rmd l}$.

        \paragraph{Otientation des lignes de champ et sens de variation de $V$.}
        On a $\vec{E}=-\vec{\text{grad}}~V$. Soit une ligne de champ (orientée selon $\vec{E}$). Alors 
        \begin{equation*}
            \vec{E}\cdot\vec{\rmd l}=-\rmd V>0,
        \end{equation*}
        donc $V$ décroît le long de la ligne de champ. On peut dire que $\vec{E}$ \og descend\fg~les potentiels.

    \subsection{Resserrement ou évasement des lignes de champ}

        C'est le principe du paratonnerre, voir la Figure~\ref{fig:resserrement_liche_champ_paratonnerre}.
        \begin{figure}
            \centering
            \tikzsetnextfilename{resserrement_liche_champ_paratonnerre}
            \begin{tikzpicture}[scale=1]  
                \draw (-2,0)--++(4,0);
                \draw (0,0)--++(0,2);
                
                \draw[->-=blue] (-1.5,4) to[bend right=10](0,2);
                \draw[->-=blue] (-0.5,4) to[bend right=5](0,2);
                \draw[->-=blue] (0.5,4) to[bend left=5](0,2);
                \draw[->-=blue] (1.5,4) to[bend left=10](0,2);
                \draw[red] (-1.5,4) to[bend left=5](1.5,4);
                \draw[red, dashed] (-1.5,4) to[bend right=5](1.5,4);
                \draw[red,-latex] (0,4)--++(0,1) node[above]{$(S_1)$};
                \draw[red,-latex] (-1,3.12)--++(-0.5,-0.4) node[above left]{$(S)$};
                \draw[red] (-0.5,2.5) to[bend left=5](0.5,2.5);
                \draw[red, dashed] (-0.5,2.5) to[bend right=5](0.5,2.5);
                \draw[red,-latex] (0.75,2)--++(-0.5,0.5) node[below right,pos=0.]{$(S_2<S_1$)};                   
                \draw[-latex] (-1,-0.5)--++(-0.5,0.5) node [below,pos=0.] {métal : $V$=constante, surface équipotentielle.};
            \end{tikzpicture}
            \caption{Resserrement des lignes de champ : principe du paratonnerre.}
            \label{fig:resserrement_liche_champ_paratonnerre}
        \end{figure}

        On a 
        \begin{equation*}
            \oiint_{(S)}\vec{E}\cdot\vec{n}^{\text{ext}}\rmd S=0=0+E_2 S_2-E_1S_1,
        \end{equation*}
        d'où
        \begin{equation*}
            \boxed{
                E_2=E_1\times\frac{S_1}{S_2}>E_1.
            }
        \end{equation*}
        Ainsi, $\left\lVert\vec{E}\right\rVert$ augmente si les lignes de champ se resserrent, et diminue si les lignes de champ s'écartent.

    \subsection{Visualisation de cartes de champ et de potentiel}
        Un exemple interactif est disponible \href{https://phyanim.sciences.univ-nantes.fr/Elec/Champs/champE.php}{sur le site de Geneviève Tulloue via l'université de Nantes}.