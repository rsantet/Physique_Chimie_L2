\section{Analogies avec la gravitation universelle}

    \subsection{Les deux lois de force. Grandeurs analogues}

    On compare dans la Table~\ref{tab:analogie_gravitation_universelle_lois_de_force} les lois et les grandeurs de l'électrostatique et de la gravitation.
    \begin{table}
        \centering
        \begin{tabular}{p{0.45\linewidth}|p{0.45\linewidth}}
            \toprule
            Électrostatique & Gravitation \\ \midrule
            $\vec{F}=\dfrac{Qq}{4\pi\varepsilon_0 r^{2}}\vec{u}_r$, attractive ou répulsive& $\vec{F}=-\mathcal{G}\dfrac{Mm}{r^{2}}\vec{u}_r$, attractive\\ \midrule
            charge (<0,>0)& masse (>0)\\ \midrule
            $\dfrac{1}{4\pi\varepsilon_0}$&$-\mathcal{G}$\\ \midrule
            Champ $\vec{E}=\dfrac{Q}{4\pi\varepsilon_0 r^{2}}\vec{u}_r$, $\vec{F}=q\vec{E}$ & Champ $\vec{g}=-\mathcal{G}\dfrac{M}{r^{2}}\vec{u}_r$, $\vec{F}=m\vec{g}$ (mais $\vec{g}$ n'est pas le champ de pesanteur)\\ \bottomrule
        \end{tabular}    
        \caption{Analogies entre électrostatique et gravitation universelle : les deux lois de force.}
        \label{tab:analogie_gravitation_universelle_lois_de_force}
    \end{table}

    \subsection[Potentiel gravitationnel]{Potentiel gravitationnel. Énergie potentielle\\rmd'une masse plongée dans un champ extérieur}

    Dans la Table~\ref{tab:analogie_gravitation_universelle_potentiel_gravitationnel}, on dérive le potentiel gravitationnel grâce à l'analogie avec l'électrostatique.
    \begin{table}
        \centering
        \begin{tabular}{p{0.45\linewidth}|p{0.45\linewidth}}
            \toprule
            Électrostatique & Gravitation \\ \midrule
            $\oint_{S}\vec{E}\cdot\vec{\rmd l}=0$ & $\oint_{C}\vec{g}\cdot\vec{\rmd l}=0$\\[0.25cm]
            $\quad\Big\Updownarrow$&$\quad\Big\Updownarrow$\\[0.25cm]
            $\exists V, \int_{A}^{B}\vec{E}\cdot\vec{\rmd l}=V(B)-V(A)$ & $\exists \varphi(\vec{r}),\int_{A}^{B}\vec{g}\cdot\vec{\rmd l}=\varphi(A)-\varphi(B)$\\[0.25cm]
            $\quad\Big\Updownarrow$&$\quad\Big\Updownarrow$\\[0.25cm]
            $\vec{E}=-\vec{\mathrm{grad}V}$ & $\vec{g}=-\vec{\mathrm{grad}}\varphi$\\[0.25cm]
            $\quad\Big\Updownarrow$&$\quad\Big\Updownarrow$\\[0.25cm]
            $\vec{\mathrm{rot}}\vec{E}=\vec{0}$ & $\vec{\mathrm{rot}}\vec{g}=\vec{0}$\\ \midrule
            Source ponctuelle (charge) & Source ponctuelle (masse)\\ \midrule
            $V(r)=\dfrac{Q}{4\pi\varepsilon_0 r}$ & $\varphi(r)=-\mathcal{G}\dfrac{M}{r}$\\ \midrule
            $E_{p}^{\text{ext}}=qV^{\text{ext}}=\dfrac{qQ}{4\pi\varepsilon_0 r}$ & $E_p^{\text{ext}}=m\varphi^{\text{ext}}=-\mathcal{G}\dfrac{mM}{r}$\\
            \bottomrule
        \end{tabular}    
        \caption{Analogies entre électrostatique et gravitation universelle : potentiel gravitationnel et énergie potentielle d'une masse plongée dans un champ extérieur.}
        \label{tab:analogie_gravitation_universelle_potentiel_gravitationnel}
    \end{table}

    \subsection{Théorème de Gauss gravitationnel}

    \begin{table}
        \centering
        \begin{tabular}{p{0.45\linewidth}|p{0.45\linewidth}}
            \toprule
            Électrostatique & Gravitation \\ \midrule
            $\oiint_{S}\vec{E}\cdot\vec{n}^{\text{ext}}\rmd S=\dfrac{Q^{\text{int}}}{\varepsilon_0}$ & $\oiint_{S}\vec{g}\cdot\vec{n}^{\text{ext}}\rmd S=-4\pi\mathcal{G}M^{\text{int}}$\\ \midrule
            Distribution sphérique & Distribution sphérique \\ \midrule
            $\vec{E}(r\geqslant R)=\dfrac{Q}{4\pi\varepsilon_0 r^{2}}\vec{u}_r$ & $\vec{g}(r\geqslant R)=-\mathcal{G}\dfrac{M}{r^{2}}\vec{u}_r$\\
            $\vec{E}(r\leqslant R)=\dfrac{Q}{4\pi\varepsilon_0 R^{2}}\dfrac{r}{R}\vec{u}_r$ &
            $\vec{g}(r\leqslant R)=-\mathcal{G}\dfrac{M}{R^{2}}\dfrac{r}{R}\vec{u}_r$\\
            (si $\rho$=constante)&(si $\mu$=constante)
            \\ \bottomrule
        \end{tabular}    
        \caption[Théorème de Gauss gravitationnel]{Analogies entre électrostatique et gravitation universelle : théorème de Gauss gravitationnel.}
        \label{tab:analogie_gravitation_universelle_theoreme_gauss_gravitationnel}
    \end{table}

    On a noté $\mu$ la masse volumique. Vu de l'extérieur, l'astre à symétrie sphérique est équivalent à une masse ponctuelle.

    \subsection{Équations locales de la gravitation universelle}

    \begin{table}
        \centering
        \begin{tabular}{p{0.45\linewidth}|p{0.45\linewidth}}
            \toprule
            Électrostatique & Gravitation \\ \midrule
            $\mathrm{div}\vec{E}=\dfrac{\rho}{\varepsilon_0}$ & $\mathrm{div}\vec{g}=-4\pi\mathcal{G}\mu$ \\ \midrule
            $\vec{\mathrm{rot}}\vec{E}=\vec{0}\Leftrightarrow\vec{E}=-\vec{\mathrm{grad}}V$ & $\vec{\mathrm{rot}}\vec{g}=\vec{0}\Leftrightarrow\vec{g}=-\vec{\mathrm{grad}}\varphi$\\ \midrule
            $\Delta V=-\dfrac{\rho}{\varepsilon_0}$ & $\Delta \varphi=4\pi\mathcal{G}\mu$
            \\ \bottomrule
        \end{tabular}    
        \caption[Équations locales de la gravitation universelle]{Analogies entre électrostatique et gravitation universelle : équations locales de la gravitation universelle.}
        \label{tab:analogie_gravitation_universelle_equations_locales_gravitation_universelle}
    \end{table}

    On a donc $\Delta \varphi=0$ entre les astres (équivalent à l'équation de Laplace).

    \subsection{Énergie potentielle de gravitation d'un astre à symétrie sphérique}

        On fait l'hypothèse que $\mu$=constante=$\frac{M}{\frac{4}{3}\pi R^{3}}$. On se demande quelle a été l'énergie de constitution de l'astre.

        L'astre s'est formé par agrégations successives de couches minces : on apporte depuis l'infini la masse $\rmd m=\mu\times 4\pi r^{2}\rmd r$ dans le potentiel 
        \begin{equation}
            \varphi(r)=-\mathcal{G}\frac{M(r)}{r}=-\mathcal{G}\frac{\mu\times\frac{4}{3}\pi r^{3}}{r}=-\mathcal{G}\frac{4\mu\pi r^{2}}{3}.
        \end{equation}

        Alors 
        \begin{align}
            \delta W_{\text{gravitationnel}}
            &=\int_{\infty}^{r}\vec{\rmd F_{\text{grav}}}\cdot\vec{\rmd l},\\
            &=\int_{\infty}^{r}\rmd m\vec{g}(s)\vec{\rmd l},\\
            &=
            \rmd m\left[
                \varphi(\infty)-\varphi(r)
            \right],\\
            &=
            -\rmd E_p,
        \end{align}
        avec $\varphi(\infty)=0$. Ainsi, on a 
        \begin{align}
            \rmd E_p
            &=
            \rmd m\varphi(r),\\
            &=
            \mu\times 4\pi r^{2}\rmd r\left(-\mathcal{G}\mu\frac{4}{3}\pi r^{2}\right),\\
            &=
            -\mathcal{G}\mu^{2}\frac{\left(4\pi\right)^{2}}{3}r^{4}\rmd r.
        \end{align}

        Ainsi, en intégrant, on a 
        \begin{equation}
            \boxed{
                E_p=\int_{0}^{R}\rmd E_p=-\mathcal{G}\mu^{2}\frac{\left(4\pi\right)^{2}}{3}\frac{R^{5}}{5}=-\frac{3\mathcal{G}M^{2}}{5R}.
            }
        \end{equation}

        Ainsi, $E_p$ diminue quand $R$ diminue : c'est l'effondrement gravitationnel. En transposant à l'électrostatique, on obtient l'énergie potentielle de constitution d'une boule chargée :
        \begin{equation}
            \boxed{
                E_p=\frac{3}{5}\frac{Q^{2}}{4\pi\varepsilon_0 R}>0.
            }
        \end{equation}