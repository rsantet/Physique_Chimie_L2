\section{Équations locales de l'électrostatique}

    \subsection{Équation de Maxwell-Gauss, théorème de Gauss}

        Le théorème de Gauss couplé ai théorème d'Ostrogradski donne
        \begin{equation}
            \oiint_{S}\vec{E}\cdot\vec{n}^{\text{ext}}\rmd S=\iiint_{V}\mathrm{div}~\vec{E}\rmd V=\frac{Q_{\text{int}}}{\varepsilon_0}=\frac{1}{\varepsilon_0}\iiint_{V}\rho\rmd V.
        \end{equation}
        Ainsi, pour tout volume $V$, on a 
        \begin{equation}
            \iiint_{V}\left(
                \mathrm{div}~\vec{E}-\frac{\rho}{\varepsilon_0}
            \right)\rmd V=0,
        \end{equation}
        d'où on entire l'équation de Maxwell-Gauss, qui est une version locale du théorème de Gauss:
        \begin{equation}
            \boxed{
                \mathrm{div}~\vec{E}(\vec{r})=\frac{\rho(\vec{r})}{\varepsilon_0}.
            }
        \end{equation}

    \subsection{Circulation conservative locale}

        On se réfère à la Figure~\ref{fig:circulation_conservative_locale_electrostatique}.

        \begin{figure}
            \centering
            \tikzsetnextfilename{circulation_conservative_locale_electrostatique}
            \begin{tikzpicture}[scale=1]  
                \node at (0,0) [below right]{O};
                \draw [->,-latex] (0,0) --++ (-0.5,-1) node [below] {x};
                \draw [->,-latex] (0,0) --++ (1.5,0) node [right]{y};
                \draw [->,-latex] (0,0) --++ (0,1.5) node [above]{z};
                \draw (0.5,0.5)--(0.75,1)--(1.75,1)--(1.5,0.5)--cycle;
                \draw[draw=green,dashed] (0.7,0.6)--(0.8,0.85)--(1.5,0.85)--(1.4,0.6)--cycle;

                \draw (3,1) circle (0.1) node [below right] {z} node {$\cdot$};
                \draw [->,-latex] (3,1) --++ (2.5,0) node [right]{y};
                \draw [->,-latex] (3,1) --++ (0,-2.5) node [below]{x};
                \draw[draw=green,->-=green] (4,0)--(4,-1.5);
                \draw[draw=green,->-=green] (4,-1.5)--(5.5,-1.5);
                \draw[draw=green,->-=green] (5.5,-1.5)--(5.5,0);
                \draw[draw=green,->-=green,text=green] (5.5,0)--(4,0) node [above,midway] {(C)};
                \draw[latex-latex,red] (4,-1.75)--(5.5,-1.75) node [below,midway] {$\rmd y$};
                \draw[latex-latex,red] (5.75,-1.5)--(5.75,0) node [right,midway] {$\rmd x$};
                
            \end{tikzpicture}
            \caption{Circulation conservative locale en coordonnées cartésiennes.}
            \label{fig:circulation_conservative_locale_electrostatique}
        \end{figure}

        On se place dans un lan parallèle à $(xOy)$, de côté $z$. On a 
        \begin{align}
            \oint_{C}\vec{E}\cdot\vec{\rmd l}
            &=0,\\
            &=
            E_x(x,y,z)\rmd x+E_y(x+\rmd x,y,z)\rmd y\\
            &\quad-E_x(x,y+\rmd y,z)\rmd x-E_y(x,y,z)\rmd y,\\
            &=
            -\rmd y\frac{\partial E_x}{\partial y}(x,y,z)\rmd x+\rmd x\frac{\partial E_y}{\partial x}(x,y,z)\rmd y=0,
        \end{align}
        d'où
        \begin{equation}
            \boxed{
            \left(\frac{\partial E_y}{\partial x}-\frac{\partial E_x}{\partial y}\right)\rmd x\rmd y=0,}
        \end{equation}
        la circulation est donc proportionnelle à l'aire du carré. De même dans les autres plans, on obtient
        \begin{equation}
            \begin{aligned}
                &\left(\frac{\partial E_x}{\partial z}-\frac{\partial E_z}{\partial x}\right)\rmd x\rmd z=0,\\
                &\left(\frac{\partial E_z}{\partial y}-\frac{\partial E_y}{\partial z}\right)\rmd y\rmd z=0.
            \end{aligned}
        \end{equation}
        La circulation du champ est donc proportionnelle à l'aire sur laquelle s'appuie le contour. L'idée est donc de passer d'une circulation $\oint\vec{E}\cdot\vec{\rmd l}$ à un flux à travers une surface.

    \subsection{Opérateur rotationnel}

        \paragraph{Théorème-définition de Stokes.}

            Soit $\vec{A}(x,y,z)$ champ de vecteurs $\mathcal{C}^{1}$. Alors on a 
            \begin{equation}
                \boxed{
                    \oint_{C}\vec{A}\cdot\vec{\rmd l}=\iint_{\Sigma}\left(\vec{\mathrm{rot}}~\vec{A}\right)\cdot\vec{N}\rmd\Sigma.
                }
            \end{equation}
            Ici, $(\Sigma)$ est une surface (non fermée) s'appuyant sur (C) et orientée selon la règle du tire-bouchon.

        \paragraph{Expression de $\vec{\mathrm{rot}}~\vec{A}$ en cartésienne.}

            On a 
            \begin{align}
                \oint_{C\in(xOy)}\vec{A}\cdot\vec{\rmd l}
                &=\left(\frac{\partial A_y}{\partial x}-\frac{\partial A_x}{\partial y}\right)\times\rmd x\rmd y,\\
                &=
                \left(\vec{\mathrm{rot}}\vec{A}\right)_{z}\rmd y\rmd y,\\
                &=
                \left(\vec{\mathrm{rot}}\vec{A}\right)\cdot\vec{u}_z\rmd x\rmd y,
            \end{align}
            donc 
            \begin{equation}
                \boxed{
                    \left(\vec{\mathrm{rot}}\vec{A}\right)_{z}=\frac{\partial A_y}{\partial x}-\frac{\partial A_x}{\partial y}.
                }
            \end{equation}
            De même pour les deux autres coordonnées, donc 
            \begin{equation}
                \boxed{
                    \vec{\mathrm{rot}}(\vec{A})=
                    \begin{pmatrix}
                        \frac{\partial A_z}{\partial y}-\frac{\partial A_y}{\partial z}\\
                        \frac{\partial A_x}{\partial z}-\frac{\partial A_z}{\partial x}\\
                        \frac{\partial A_y}{\partial x}-\frac{\partial A_x}{\partial y}
                    \end{pmatrix}.
                }
            \end{equation}
            Avec l'opérateur nabla (en cartésienne), on a 
            \begin{equation}
                \boxed{
                    \vec{\nabla}\wedge\vec{A}=\begin{pmatrix}
                        \frac{\partial}{\partial x}\\
                        \frac{\partial}{\partial y}\\
                        \frac{\partial}{\partial z}
                    \end{pmatrix}\wedge
                    \begin{pmatrix}
                        A_x\\A_y\\A_z
                    \end{pmatrix}=\vec{\mathrm{rot}}\vec{A}
                }.
            \end{equation}

    \subsection{Équation de Maxwell--Faraday}

        Sur un contour fermé $C$ avec une surface $\Sigma$ supportée par ce contour, on a 
        \begin{equation}
            \oint_{C}\vec{E}\cdot\vec{\rmd l}=0=\iint_{\Sigma}\vec{\mathrm{rot}}\vec{E}\rmd\Sigma,
        \end{equation}
        ceci vaut pour tout contour $C$, d'où on en tire l'équation de Maxwell-Faraday, qui ets une équation locale :
        \begin{equation}
            \boxed{
                \vec{\mathrm{rot}}\vec{E}(\vec{r})=\vec{0}.
            }
        \end{equation}

        \paragraph{Lien avec le potentiel.}

            Soit $f(\vec{r})=\vec{\mathrm{rot}}(\vec{\mathrm{grad}}f)=\vec{0}$ car
            \begin{equation}
                \oint_{C}\vec{\mathrm{grad}}f\cdot\vec{\rmd l}=0=\iint_{\Sigma}\vec{\mathrm{rot}}(\vec{\mathrm{grad}}f)\cdot\vec{N}\rmd\Sigma,
            \end{equation}
            donc
            \begin{equation}
                \boxed{
                    \vec{\mathrm{rot}}\vec{E}=\vec{0}\Longleftrightarrow\vec{E}=-\vec{\mathrm{grad}}V.
                }
            \end{equation}

    \subsection{Vue d'ensemble des différentes formulations des lois de l'électrostatique}

        On choisit un des points de vue, qui sont tous équivalents.
        \begin{enumerate}[align=left]
            \item [\underline{Loi fondamentale}]\phantom{}
            \begin{itemize}
                \item Loi de Coulomb : $\vec{E}=\dfrac{Q}{4\pi\varepsilon_0 r^{2}}\vec{u}_r$
                \item Principe de superposition
            \end{itemize}
            \item [\underline{Formulation intégrale}]\phantom{}
            \begin{itemize}
                \item Théorème de Gauss : $\oiint_{S}\vec{E}\cdot\vec{n}^{\text{ext}}\rmd S=\frac{Q^{\text{int}}}{\varepsilon_0}$
                \item Circulation conservative de $\vec{E}$ : $\oint\vec{E}\cdot\vec{\rmd l}=0\Leftrightarrow\int_{A}^{B}\vec{E}\cdot\vec{\rmd l}=V(A)-V(B)$
            \end{itemize}
            \item [\underline{Formulation locale}]\phantom{}
            \begin{itemize}
                \item Maxwell-Gauss : $\mathrm{div}\vec{E}=\frac{\rho}{\varepsilon_0}$ (flux de $\vec{E}$)
                \item Maxwell-Faraday : $\vec{\mathrm{rot}}\vec{E}=\vec{0}\Leftrightarrow\vec{E}=-\vec{\mathrm{grad}}V$ (circulation de $\vec{E}$)
            \end{itemize}
        \end{enumerate}
        