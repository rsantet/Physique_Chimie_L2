\section{Équations de Poisson et de Laplace}

    \subsection{Équation de Poisson en présence de charges}

        On a $\div\vec{E}=\frac{\rho}{\varepsilon_0}$ et $\vec{E}=-\vecgrad V$, d'où $-\div(\vecgrad V)=\frac{\rho}{\varepsilon_0}$. En coordonnées cartésiennes, on a 
        \begin{equation}
            \vec{\nabla}\cdot(\vec{\nabla}V)=\frac{\partial}{\partial x}\left(\frac{\partial V}{\partial x}\right)+\frac{\partial}{\partial y}\left(\frac{\partial V}{\partial y}\right)+\frac{\partial}{\partial z}\left(\frac{\partial V}{\partial z}\right),
        \end{equation}
        et donc 
        \begin{equation}
            \boxed{
                \div\left(\vecgrad V\right)=\frac{\partial^{2}V}{\partial x^{2}}+\frac{\partial^{2}V}{\partial y^{2}}+\frac{\partial^{2}V}{\partial z^{2}}\coloneqq\Delta V.
            }
        \end{equation}
        L'équation de Poisson (locale) s'écrit donc
        \begin{equation}
            \boxed{
                \Delta V(x,y,z)=-\frac{\rho(x,y,z)}{\varepsilon_0}.
            }
        \end{equation}
        Cette équation contient \og toute l'électrostatique\fg.

    \subsection{Équation de Laplace. Résolution numérique}

        Dans le vide, en l'absence de charges, on a 
        \begin{equation}
            \boxed{
                \Delta V(x,y,z)=0.
            }
        \end{equation}
        C'est l'équation de Laplace. 

        On a le théorème d'unicité suivant : pour des conditions aux limites données (sur $V$ ou ses dérivées), et pour une géométrie donnée, l'équation de Laplace admet une unique solution. On donne le principe de la résolution numérique de $\Delta V=0$ en deux dimensions sur un domaine maillé uniforme de pas constant $l$. On note $V_{n,m}=V(nl,ml)$.
        Alors
        \begin{equation}
            \begin{aligned}
                V_{n-1,m}&= V_{n,m}-l\left(\frac{\partial V}{\partial x}\right)_{n,m}+\frac{l^{2}}{2}\left(\frac{\partial^{2}V}{\partial x^{2}}\right)_{n,m}+o(l^{2}),\\
                V_{n+1,m}&= V_{n,m}+l\left(\frac{\partial V}{\partial x}\right)_{n,m}+\frac{l^{2}}{2}\left(\frac{\partial^{2}V}{\partial x^{2}}\right)_{n,m}+o(l^{2}),\\
                V_{n,m-1}&= V_{n,m}-l\left(\frac{\partial V}{\partial y}\right)_{n,m}+\frac{l^{2}}{2}\left(\frac{\partial^{2}V}{\partial y^{2}}\right)_{n,m}+o(l^{2}),\\
                V_{n,m+1}&= V_{n,m}+l\left(\frac{\partial V}{\partial y}\right)_{n,m}+\frac{l^{2}}{2}\left(\frac{\partial^{2}V}{\partial y^{2}}\right)_{n,m}+o(l^{2}),\\
            \end{aligned}
        \end{equation}
        Comme $\Delta V=0$, la somme donne
        \begin{equation}
            \boxed{
                V_{n,m}=\frac{V_{n-1,m}+V_{n+1,m}+V_{n,m-1}+V_{n,m+1}}{4},
            }
        \end{equation}
        que l'on résout avec un tableur.