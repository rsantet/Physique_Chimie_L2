\section{Flux du champ électrostatique. Théorème de Gauss}

    \subsection{Charge ponctuelle : flux à travers une sphère.}

        On considère une charge ponctuelle $Q$ en un point $O$ et une sphère $S$ de rayon $r$ de centre $O$. On note
        \begin{equation*}
            \vec{E}=\frac{Q}{4\pi\varepsilon_{0}r^{2}}\vec{u_r}.
        \end{equation*}
        Alors en notant $\vec{n}^{\text{ext}}$ le vecteur normal à la surface $S$,
        \begin{equation*}
            \boxed{
                \oiint_{S}\vec{E}\cdot\vec{n}^{\text{ext}}\d S=\frac{Q}{4\pi\varepsilon_{0}r^{2}}\oiint\d S=\frac{Q}{\varepsilon_{0}}.
            }
        \end{equation*}

    \subsection{Théorème de Gauss}
        \subsubsection{Une charge ponctuelle à l'intérieur d'une surface fermée}

            Soit $V$ un volume quelconque de l'espace contenant une charge $Q$. On note $S$ une sphère centrée en $Q$ contenue dans $V$, et $S'$ le reste de la surface correspondant à $V$.En un point $M$ quelconque du volume, on a
            \begin{equation*}
                \vec{E}(M)=\frac{Q}{4\pi\varepsilon_{0}r^{2}}\vec{u_r}.
            \end{equation*}

            On a 
            \begin{equation*}
                \oiint_{S\cup S'}\vec{E}\cdot\vec{n}^{\text{ext}}\d S=\iiint_{V}\div\vec{E}\d V,
            \end{equation*}
            et pour un problème à symétrie sphérique, on a pour tout $r\neq0$,
            \begin{equation*}
                \div\vec{E}=\frac{1}{r^{2}}\frac{\partial}{\partial r}\left(r^{2}E_r\right).
            \end{equation*}
            Ainsi, $\div\vec{E}=0$ pour tout $r\neq 0$, d'où 
            \begin{align*}
                \iiint_{V}\div\vec{E}\d V=0=\oiint_{S}\vec{E}\cdot\vec{n}^{\text{ext}}\d S+\oiint_{S'}\vec{E}\cdot\vec{n}^{\text{ext}}\d S',
            \end{align*}
            et les deux normales extérieures sont opposées l'une de l'autre. Finalement,
            \begin{equation*}
                \oiint_{S}\vec{E}\cdot\vec{n}^{\text{ext}}\d S=\oiint_{S}\vec{E}\cdot\vec{u_r}\d S'=\frac{Q}{\varepsilon_{0}}.
            \end{equation*}

            Ainsi, si une surface $S$ contient une charge $Q$, on a toujours
            \begin{equation*}
                \boxed{
                    \oiint_{S}\vec{E}\cdot\vec{n}^{\text{ext}}\d S=\frac{Q}{\varepsilon_{0}}.
                }
            \end{equation*}

        \subsubsection{Charge ponctuelle à l'extérieur d'une surface fermée}

            Dans ce cas, on a directement
            \begin{equation*}
                \boxed{
                    \oiint_{S}\vec{E}\cdot\vec{n}^{\text{ext}}\d S=\iiint_{V}\div\vec{E}\d V=0.
                }
            \end{equation*}

        \subsubsection{Bilan}

            Si $\vec{E}$ est le champ total créé par $N$ charges $Q_i$, alors
            \begin{equation*}
                \boxed{
                    \oiint_{S}\vec{E}\cdot\vec{n}^{\text{ext}}\d S=\frac{Q^{\text{int}}}{\varepsilon_{0}}.
                }
            \end{equation*}

            Ici, $Q^{\text{int}}$ est la somme de toutes les charges qui sont à l'intérieur de la surface $S$.

    \subsection{Quand et comment appliquer le théorème de Gauss}

        Le but est de calculer des champs électrostatiques $\vec{E}$ dans des cas de hautes symétries. La méthode est la suivante:
        \begin{itemize}
            \item [($\alpha$)] Invariance et symétries : donne la géométrie de $\vec{E}$;
            \item [($\beta$)] Choisir une surface de Gauss adaptée (ou bien $\vec{E}\parallel\vec{n}^{\text{ext}}$ avec $E=$constante sur la surface ou bien $\vec{E}\perp\vec{n}^{\text{ext}}$) avec un dessin;
            \item [($\gamma$)] Conclure.
        \end{itemize}

    \subsection{Exemples fondamentaux}
        \subsubsection{Sphère uniformément chargée en volume ou en surface}

            On considère le système donnée à la Figure~\ref{fig:sphere_uniformement_chargee_volume_surface}.
            \begin{figure}
                \centering
                \tikzsetnextfilename{sphere_uniformement_chargee_volume_surface}
                \begin{tikzpicture}[scale=1]  
                    % \helpgrid{3}{3}
                    \coordinate (O) at (0,0);
                    \node at (O) {$\bullet$};
                    \node at (O) [left] {O};
                    \coordinate (A) at (2,2);
                    \draw [->, -stealth] (O)--(0,2.5) node [above] {z};
                    \draw[dashed] (O)--(A) node [below, midway, shift={(0.2,-0.1)}] {$r$};
                    \node at (A) {$\bullet$};
                    \node at (A) [above] {M};
                    \draw [->] (0,0.75) to [bend left=45] (0.5,0.5) node [above] {$\theta$};
                    \draw [->] (-0.25,2.3) to [bend right=90] (0.25,2.3) node [right] {$\varphi$};
                    \draw [pattern=north east lines, opacity=0.25] (O) circle (1.5);
                    \draw[<->] (O)--(-45:1.5) node [below, midway] {$R$};
                \end{tikzpicture}
                \caption{Sphère uniformément chargée en volume ou en surface.}    
                \label{fig:sphere_uniformement_chargee_volume_surface}
            \end{figure}

            \begin{itemize}
                \item [($\alpha$)] Toute rotation d'axe passant par O laisse la distribution invariant, donc 
                \begin{equation*}
                    \vec{E}(r,\theta,\varphi)=\vec{E}(r)=\begin{pmatrix}
                        E_r(r)\\ E_{\theta}(r)\\E_{\varphi}(r)
                    \end{pmatrix}.
                \end{equation*}
                Les plans $(M,\vec{u_r},\vec{u_{\varphi}})$ et $(M,\vec{u_r},\vec{u_{\theta}})$ sont des PS valable pour tout point $M$. Ainsi, $\vec{E}$ est radial et $\vec{E}(M)=E(r)\vec{u_r}$.
                \item [($\beta$)] La bonne surface de Gauss est une sphère de centre $O$ et de rayon $r$ variable.
                \item [($\gamma$)] On a 
                \begin{equation*}
                    \oiint_{S}\vec{E}\cdot\vec{n}^{\text{ext}}\d R=\oiint_{S}E(r)\d S=E(r)4\pi r^{2}=\frac{Q_{\text{int}}(r)}{\varepsilon_{0}}.
                \end{equation*}
                Ainsi,
                \begin{equation*}
                    \boxed{
                        \vec{E}(r)=\frac{Q_{\text{int}}(r)}{4\pi\varepsilon_{0}r^{2}}\vec{u_r}.
                    }
                \end{equation*}
            \end{itemize}

            \paragraph{Boule uniformément chargée en volume.}

                La densité $\rho$ est constante et vaut $\frac{Q}{\frac{4}{3}\pi R^{3}}$ où $Q$ est la charge présente dans l'entièreté de la boule. Ainsi, pour $r\geqslant R$, on a 
                \begin{equation*}
                    Q_{\text{int}}(r\geqslant R)=Q,
                \end{equation*}
                d'où
                \begin{equation*}
                    \boxed{
                        \vec{E}(r\geqslant R)=\frac{Q}{4\pi\varepsilon_0 r^{2}}\vec{u_r}.
                    }
                \end{equation*}

                Si $r\leqslant R$, alors on a 
                \begin{equation*}
                    Q_{\text{int}}(r\leqslant R)=\iiint \rho\d V=\rho\frac{4}{3}\pi r^{3}=Q\left(\frac{r}{R}\right)^{3}.
                \end{equation*}
                Ainsi, on a 
                \begin{equation*}
                    \boxed{
                        \vec{E}(r\leqslant R)=\frac{Qr}{4\pi\varepsilon_0 R^{3}}\vec{u_r}=\frac{Q}{4\pi\varepsilon_0 R^{2}}\left(\frac{r}{R}\right)\vec{u_r}.
                    }
                \end{equation*}

            \paragraph{Sphère uniformément chargée en surface.}

                Pour $r>R$, on a le même résultat que précédemment. Pour $r<R$, on a $Q_{\text{int}}(r<R)=0$, donc $\vec{E}(r<R)=\vec{0}$. $\vec{E}$ est donc discontinu à la surface de la sphère (variation d'amplitude égal à $\sigma/\varepsilon_{0}$). C'est un modèle non physique.                

        \subsubsection{Cylindre infini uniformément chargé en volume ou en surface}

            On considère un cylindre de rayon $R$ d'axe $(Oz)$.
            La distribution de charge est à symétrie cylindrique \og infinie\fg. 

            \begin{itemize}
                \item [($\alpha$)] Il y a une symétrie de révolution par rapport à l'axe $(Oz)$ et une invariance par translation parallèlement à l'axe $(Oz)$. De plus, tout plan perpendiculaire à $(Oz)$ est un plan de symétrie. Enfin, tout plan contenant $(Oz)$ est un plan de symétrie. Finalement, on a 
                \begin{equation*}
                    \boxed{
                    \vec{E}(M)=E(r)\vec{u_r}.
                    }
                \end{equation*}
                \begin{remark}
                    Si la distribution est non infinie, a priori on a 
                    \begin{equation*}
                        \vec{E}(M)=\begin{pmatrix}
                            E_r(r,z)\\0\\e_z(r,z)
                        \end{pmatrix}.
                    \end{equation*}
                \end{remark}

                \item [$(\beta$)] La surface de Gauss que l'on prend est un cylindre d'axe $(Oz)$ de rayon $r$, de hauteur de $h$, formé par deux disques perpendiculaires à l'axe $(Oz)$.
                
                \item [$(\gamma)$] Le théorème de Gauss donne
                \begin{equation*}
                    \oiint_{S}\vec{E}\cdot\vec{n}^{\text{ext}}\d S=\frac{Q_{\text{int}}(r)}{\varepsilon_{0}},
                \end{equation*}
                où $S=\Sigma\cup S_1\cup S_2$ où $S_1$ et $S_2$ correspondent aux disques. Ainsi, 
                \begin{equation*}
                    \oiint_{S}\vec{E}\cdot\vec{n}^{\text{ext}}\d S=0+0+\iint_{\Sigma}E(r)\vec{u_r}\cdot\vec{n}^{\text{ext}}\d \Sigma=E(r)\times 2\pi r h.
                \end{equation*}
                Ainsi,
                \begin{equation*}
                    \boxed{
                        E(r)=\frac{Q_{\text{int}}(r)}{2\pi\varepsilon_{0}hr}.
                    }
                \end{equation*}

            \end{itemize}

            \paragraph{Cylindre uniformément charge en volume.} 
            
                On considère une tranche d'hauteur $h$, on a $Q=\rho h\pi R^{2}$. On introduit donc 
                \begin{equation*}
                    \boxed{
                        \lambda=\frac{Q}{h}=\rho\pi R^{2}.
                    }
                \end{equation*}
                C'est la charge linéique en \si{\coulomb\per\metre}. Pour $r\geqslant R$, on a $Q_{\text{int}}(r\geqslant R)=\lambda h$, d'où 
                \begin{equation*}
                    \boxed{
                        \vec{E}(r\geqslant R)=\frac{\lambda}{2\pi\varepsilon_{0}r}\vec{u_r}.
                    }
                \end{equation*}

                Si $r\leqslant R$, on a $Q_{\text{int}}(r\leqslant R)=\rho \pi r^{2}h=\lambda h\left(\frac{r}{R}\right)^{2}$. Ainsi,
                \begin{equation*}
                    \boxed{
                        \vec{E}(r\leqslant R)=\frac{\lambda r}{2\pi\varepsilon_{0}R^{2}}\vec{u_r}.
                    }
                \end{equation*}

            \paragraph{Cylindre uniformément chargé en surface.} 

                Si $r<R$, on a $\vec{E}(r<R)=\vec{0}$. Si $r>R$, on a $Q_{\text{int}}(r>R)=Q=\lambda h$, donc $\vec{E}(r>R)=\frac{\lambda}{2\pi\varepsilon_{0}r}\vec{u_r}$. À nouveau, il y a une discontinuité égale à $\sigma/@\varepsilon_{0}$ avec $\sigma=\frac{\lambda}{2\pi R}$. Elle est due au modèle surfacique.

            
        \subsubsection{Plan infini uniformément chargé en surface}

            On considère le système décrit à la Figure~\ref{fig:plan_uniformement_charge_surface}.
            \begin{figure}
                \centering
                \tikzsetnextfilename{plan_uniformement_charge_surface}
                \begin{tikzpicture}[scale=1]  
                    % \helpgrid{3}{3}
                    \coordinate (A) at (-2,-0.5);
                    \coordinate (B) at (-1.5,0);
                    \node at (0,0) [below right]{O};
                    \draw [->,-latex] (0,0) --++ (-0.5,-1) node [below] {x};
                    \draw [->,-latex] (0,0) --++ (1.5,0) node [right]{y};
                    \draw [->,-latex] (0,0) --++ (0,1.5) node [above]{z};

                    \draw (-4,-1) -- (-3,1) -- (5,1) -- (4,-1) -- cycle;

                    \node at (3,-0.5) {$\sigma$};
                  
                    \filldraw [fill=gray!20, opacity=0.4] (A) --++ (1,0) --++ (0,2) --++ (-1,0) -- cycle;
                    \filldraw [fill=gray!30, opacity=0.4] (B) --++ (1,0) --++ (0,2) --++ (-1,0) -- cycle;
                    \draw[opacity=0.4] (A) -- (B);
                    \draw[opacity=0.4] ($(A)+(1,0)$) -- ($(B)+(1,0)$);
                    \draw[opacity=0.4] ($(A)+(1,2)$) -- ($(B)+(1,2)$);
                    \draw[opacity=0.4] ($(A)+(0,2)$) -- ($(B)+(0,2)$);

                    \filldraw [fill=gray!20, opacity=0.4,dashed] (A) --++ (1,0) --++ (0,-2) --++ (-1,0) -- cycle;
                    \filldraw [fill=gray!30, opacity=0.4, dashed] (B) --++ (1,0) --++ (0,-2) --++ (-1,0) -- cycle;
                    \draw[opacity=0.4, dashed] ($(A)+(1,-2)$) -- ($(B)+(1,-2)$);
                    \draw[opacity=0.4, dashed] ($(A)+(0,-2)$) -- ($(B)+(0,-2)$);

                    \fill[pattern=north east lines, pattern color=blue] (A)--++(1,0)--($(B)+(1,0)$)--(B)--cycle;
                    \node[text=blue] at ($(A)+(-0.5,0)$) {$\sigma_S$};
                \end{tikzpicture}
                \caption{Plan uniformément chargé en surface.}    
                \label{fig:plan_uniformement_charge_surface}
            \end{figure}

            Par plan infini, on entend que les longueurs caractéristiques du plan selon les axes $x$ et $y$ sont très grandes devant l'épaisseur du plan : $L_x,L_y\gg e$.

            \begin{itemize}
                \item [($\alpha$)] Il y a invariance par translation par rapport aux axes $(Ox)$ et $(Py)$. Ainsi, le champ ne dépend pas de $x$ ni de $y$. Pour les symétries, tout plan parallèle à $(xOy)$ est un PS, donc $E_y=0$. Tout plan parallèle à $(yOz)$ est un PS, donc $E_x=0$. Ainsi, on a $\vec{E}(M)=E(z)\vec{u}_z$.
                Enfin, le fait que la plan $(xOy)$ est un PS implique sur $E(-z)=E(z)$.

                \item [($\beta$)] La surface de Gauss est un cylindre de générateur parallèle à $(Oz)$, de hauteur 2z centré sur le plan $z=0$.
                
                \item [($\gamma$)] On a 
                \begin{align*}
                    \oiint_{S=S_1\cup S_2\cup \Sigma}\vec{E}\cdot\vec{n}^{\text{ext}}\d S
                    &=\frac{Q_{\text{int}}}{\varepsilon_0},\\
                    &=\frac{\sigma S}{\varepsilon_0},\\
                    &=0+\iint_{S_1}E(z)\vec{u}_z\cdot\vec{u}_z\d S+\iint_{S_2}E(-z)\vec{u}_z\cdot\left(-\vec{u}_z\right)\d S,\\
                    &=
                    E(z)S+E(z)S.
                \end{align*}
                Ainsi, 
                \begin{equation*}
                    \vec{E}(z>0)=\frac{\sigma}{2\varepsilon_0}\vec{u}_z,
                \end{equation*}
                et 
                \begin{equation*}
                    \vec{E}(z<0)=-\frac{\sigma}{2\varepsilon_0}\vec{u}_z.
                \end{equation*}
            \end{itemize}