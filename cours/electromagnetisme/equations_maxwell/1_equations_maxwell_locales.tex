\section{Équations de Maxwell locales}

Elles régissent le champ électromagnétique. Il y a d'abord les équations de structure:
\begin{equation}
    \boxed{
        \begin{aligned}
            \div\vec{B}(\vec{r},t)&=0&\text{Maxwell-Thompson},\\
            \vecrot\vec{E}(\vec{r},t)&=-\frac{\partial\vec{B}}{\partial t}(\vec{r},t)&\text{Maxwell-Faraday}.
        \end{aligned}
    }
\end{equation}
Il y a ensuite le lien avec les sources:
\begin{equation}
    \boxed{
        \begin{aligned}
            \div\vec{E}(\vec{r},t)&=\frac{\rho(\vec{r},t)}{\varepsilon_0}&\text{Maxwell-Gauss},\\
            \vecrot\vec{B}(\vec{r},t)&=\mu_0\vec{j}(\vec{r},t)+\frac{1}{c^{2}}\frac{\partial\vec{E}(\vec{r},t)}{\partial t}&\text{Maxwell-Ampère}.
        \end{aligned}
    }
\end{equation}
En plus de l'expression de la force de Lorentz
\begin{equation}
    \boxed{
        \vec{F}_L=q(\vec{E}+\vec{v}\wedge\vec{B}),
    }
\end{equation}
on a condensé tout l'électromagnétisme. Il faut remarquer que:
\begin{itemize}
    \item il y a un couplage spatio-temporel de $\vec{E}$ et $\vec{B}$ via Maxwell-Faraday et Maxwell-Ampère;
    \item Maxwell-Gauss et Maxwell-Thompson sont toujours vraies;
    \item il apparaît une constant $c$ telle que $\varepsilon_0\mu_0 c^{2}=1$ (invariance par changement de référentiel galiléen), c'est le cadre naturel de l'électromagnétisme est relativiste;
    \item les équations sont linéaires, on peut donc utiliser le théorème de superposition.
\end{itemize}