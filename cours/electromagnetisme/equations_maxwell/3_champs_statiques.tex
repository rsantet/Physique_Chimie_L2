\section{Cas des champs statiques}
\subsection{Électrostatique}

En régime permanent, on a $\div\vec{B}(\vec{r})=0$ et $\vecrot\vec{B}(\vec{r})=\vec{0}$ pour tout $\vec{r}$, on $\vec{B}(\vec{r})=\vec{0}$. Ainsi,
\begin{equation}
    \boxed{
        \left\lbrace
        \begin{aligned}
            \div\vec{E}(\vec{r})&=\frac{\rho(\vec{r})}{\varepsilon_0},\\
            \vecrot\vec{E}(\vec{r})&=\vec{0}.
        \end{aligned}
        \right.
    }
\end{equation}
Ainsi $\vec{E}$ est à circulation conservative.

\subsection{Magnétostatique}

En régime permanent, on a aussi $\div\vec{E}(\vec{r})=0$ et $\vecrot\vec{E}(\vec{r})=\vec{0}$, d'où
\begin{equation}
    \boxed{
        \left\lbrace
        \begin{aligned}
            \div\vec{B}(\vec{r})&=0,\\
            \vecrot\vec{B}(\vec{r})&=\mu_0\vec{j}(\vec{r}).
        \end{aligned}
        \right.
    }
\end{equation}