\section[Équation locale de Poynting]{Équation locale de Poynting : conservation de l'énergie électromagnétique}

On se place dans un cas général (pas nécessairement un conducteur ohmique).

\subsection{Équation de Poynting}

L'idée est d'avoir une équation locale traduisant les échanges d'énergie champ/matière.
On part de
\begin{equation}
    \vec{j}\cdot\vec{E}=\vec{E}\cdot\left(
        \vec{\rot}\left(\frac{\vec{B}}{\mu_0}\right)-\varepsilon_0\frac{\partial\vec{E}}{\partial t}
    \right).
\end{equation}
On se donne la formule suivante:
\begin{equation}
    \div\left(\vec{C}\wedge\vec{D}\right)=\vec{D}\cdot\vec{\rot}\vec{C}-\vec{C}\cdot\vec{\rot}\vec{D}.
\end{equation}
On obtient alors
\begin{equation}
    \vec{j}\cdot\vec{E}=\vec{E}\cdot\left(\vec{\rot}\left(\frac{\vec{B}}{\mu_0}\right)-\varepsilon_0\frac{\partial\vec{E}}{\partial t}\right),
\end{equation}
puis
\begin{equation}
    \boxed{
        \frac{\partial}{\partial t}\left(
            \frac{\varepsilon_0 E^{2}}{2}+\frac{B^{2}}{2\mu_0}
        \right)+\div\left(\vec{E}\wedge\frac{\vec{B}}{\mu_0}\right)=-\vec{j}\cdot\vec{E}.
    }
\end{equation}

\subsection{Densité volumique d'énergie électromagnétique et vecteur de Poynting}

On a le postulat suivant:
\begin{quote}
    Là où est le champ, là est l'énergie.
\end{quote}
Ainsi, l'énergie électromagnétique est localisée, non dans les sources [$\rho,\vec{j}$] ou dans la matière, mais dans le champ électromagnétique [$\vec{E},\vec{B}$].

On définit la densité volumique d'énergie électromagnétique par 
\begin{equation}
    \boxed{
        u_{\mathrm{em}}(\vec{r},t)\coloneqq\frac{\varepsilon_0 E^{2}(\vec{r},t)}{2}+\frac{B^{2}(\vec{r},t)}{2\mu_0}.
    }
\end{equation}
Son unité est le $\si{\joule\per\metre\cubed}$.

On définit le vecteur de Poynting, qui correspond au vecteur densité de courant d'énergie électromagnétique, par
\begin{equation}
    \boxed{
        \vec{\Pi}(\vec{r},t)\coloneqq\vec{E}(\vec{r},t)\wedge\frac{\vec{B}(\vec{r},t)}{\mu_0}.
    }
\end{equation}
Son unité est le $\si{\watt\per\metre\squared}$. Alors on a l'équation locale de conservation de l'énergie électromagnétique suivante:
\begin{equation}
    \boxed{
        \frac{\partial u_{\mathrm{em}}}{\partial t}+\div\vec{\Pi}=-\vec{j}\cdot\vec{E}=-P_{\mathrm{vol}}.
    }
\end{equation}
Du point de vue du champ électromagnétique, si $P_{\mathrm{vol}}>0$ (conducteur ohmique) alors c'est un terme \og puits\fg, et si $P_{\mathrm{vol}}<0$ (milieu amplificateur comme dans un LASER), alors c'est un terme \og source\fg.

La puissance électromagnétique sortant d'une surface orientée $S$ est donnée par
\begin{equation}
    \boxed{
        P_{S}^{\mathrm{em}}=\iint_{S}\vec{\Pi}\cdot\vec{n}\d S,
    }
\end{equation}

\subsection{Ordres de grandeurs}
\subsubsection{LASER Helium-Néon}
$\left\langle P\right\rangle\approx1\,\si{\milli\watt}$ et $S\approx1\,\si{\milli\metre}$, donc $\left\langle\Pi\right\rangle\approx1\,\si{\kilo\watt\per\metre\squared}$.

\subsubsection{Soleil sur la Terre}
$\varphi_{\mathrm{sol}}\sim1\,\si{\kilo\watt\per\metre\squared}=\left\langle\Pi\right\rangle$.

\subsubsection{Téléphone portable}
En mode \og passif\fg, on a $\left\langle\Pi\right\rangle\sim10\,\si{\milli\watt\per\metre\squared}$. En mode \og actif\fg, on a $\left\langle\Pi\right\rangle\sim100\,\si{\milli\watt\per\metre\squared}$.

\subsection{Propagation unidimensionnelle dans un milieu absorbant. Loi de Beer-Lambert}

On se place dans le cas de la Figure~\ref{fig:propagation_unidimensionnelle_milieu_absorbant}. On fait l'hypothèse que l'on a
\begin{equation}
    \left\langle P_{\mathrm{vol}}(x)\right\rangle=\alpha\left\langle\left\lVert\vec{\Pi}\right\rVert\right\rangle(x),
\end{equation}
avec $\alpha>0$. L'équation de Poynting donne alors
\begin{equation}
    \left\langle\frac{\partial u_{\mathrm{em}}}{\partial t}\right\rangle+\frac{\partial\left\langle\Pi\right\rangle}{\partial x}=-\left\langle P_{\mathrm{vol}}\right\rangle.
\end{equation}

Remarquons alors que pour une fonction $f$ $T$-périodique, on a
\begin{equation}
    \left\langle\frac{\partial f^{2}}{\partial t}\right\rangle=\frac{1}{T}\int\frac{\partial f^{2}}{\partial t}\d t=\frac{1}{T}(f^{2}(t_0+T)-f^{2}(t_0))=0.
\end{equation}
Ainsi, on obtient
\begin{equation}
    \frac{\d\left\langle\Pi\right\rangle}{\d x}=-\alpha\left\langle\Pi\right\rangle,
\end{equation}
car $\left\langle\frac{\partial u_{\mathrm{em}}}{\partial t}\right\rangle=0$ ($\vec{E}$ et $\vec{B}$ sont périodiques). Donc $\left\langle \Pi\right\rangle(x)=\left\langle\Pi\right\rangle_{0}\e^{-\alpha x}$. On a alors 
\begin{equation}
    \left\langle\Pi\right\rangle_{\mathrm{out}}=\left\langle\Pi\right\rangle_{\mathrm{in}}\e^{-\alpha l},
\end{equation}
et donc 
\begin{equation}
    \ln\left(
        \frac{\left\langle\Pi\right\rangle_{\mathrm{in}}}{\left\langle\Pi\right\rangle_{\mathrm{out}}}
    \right)=\alpha l.
\end{equation}
Le terme de gauche est appelée l'absorbance $A$, et $\alpha$ dépend de la longueur d'onde $\lambda$ de la lumière émise et du type de la solution, et on obtient la loi de Beer-Lambert:
\begin{equation}
    \boxed{
        A(\lambda,\mathrm{sol},l)=\alpha(\lambda,\mathrm{sol})l.
    }
\end{equation}

\begin{figure}
    \centering
    \tikzsetnextfilename{propagation_unidimensionnelle_milieu_absorbant}
    \begin{tikzpicture}[scale=1]  
        % \helpgrid{3}{3}
        \draw[->] (0,0) --++ (4,0) node [right] {$x$};
        \draw (0,0) --++(0,2);
        \draw (2,0) --++(0,2);
        \draw[dashed, ->] (0,1) --++(3,0) node [pos=0, left] {lumière};
        \draw[draw=blue,text=blue] (0.75,0) --++ (0,2) node [pos=0,below] {\tiny $x$};
        \draw[draw=blue,text=blue] (1.25,0) --++ (0,2) node [pos=0,below] {\tiny $x+\d x$};
        \draw[<->] (0,-0.5)--++(2,0) node [midway,below] {$l$};
    \end{tikzpicture}
    \caption{Propagation unidimensionnelle dans un milieu absorbant.}    
    \label{fig:propagation_unidimensionnelle_milieu_absorbant}
\end{figure}