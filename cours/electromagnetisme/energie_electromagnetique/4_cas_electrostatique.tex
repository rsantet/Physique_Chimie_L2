\section{Cas de l'électrostatique}

On a $\vec{B}=\vec{0}$ et donc $\vec{\Pi}=\vec{0}$.

\subsection{Énergie du champ électrostatique}

On a 
\begin{equation}
    \boxed{
        u_{\mathrm{em}}(\vec{r},t)=\frac{\varepsilon_0 E^{2}(\vec{r},t)}{2}+0.
    }
\end{equation}
L'énergie du champ créé par la distribution de charges créant le champ électrostatique est alors 
\begin{equation}
    \boxed{
        U_{\mathrm{es}}=\iiint_{\text{tout l'espace}}\frac{\varepsilon_0 E^{2}(\vec{r})}{2}\d\tau.
    }
\end{equation}

\subsection{Définition énergétique de la capacité\texorpdfstring{\\}{ }d'un condensateur}

On considère que $\vec{E}$ est localisé entre les armatures. Par exemple, dans un condensateur plan sans effets de bords, on a $\vec{E}=-U/e\vec{u}_z$ avec $e$ la distance entre les deux armatures, disposées perpendiculairement par rapport à l'axe $\vec{u}_z$ et $U$ est la tension appliquée entre les deux armatures. Le champ est nul en dehors des armatures. Alors
\begin{align}
    U_{\mathrm{es}}
    &=
    \iiint_{\text{tout l'espace}}\frac{\varepsilon_0 E^{2}}{2}\d\tau\\
    &=
    \iiint_{\substack{\text{espaces}\\\text{inter-armatures}}}\frac{\varepsilon_0 E^{2}}{2}\d\tau\\
    &=
    \frac{\varepsilon_0}{2}\frac{U^{2}}{e^{2}}\times(e S)=\frac{1}{2}\frac{\varepsilon_0 S}{e}U^{2}\coloneqq\frac{1}{2}C U^{2}.
\end{align}

La définition générale est donnée par
\begin{equation}
    \boxed{
        \frac{1}{2}CU^{2}=\frac{1}{2}\frac{Q^{2}}{C}=\iiint_{\text{tout l'espace}}\frac{\varepsilon_0 E^{2}}{2}\d\tau.
    }
\end{equation}