\section{Bilan intégral\texorpdfstring{\\}{ }d'énergie électromagnétique}
\subsection{Bilan intégral sur un volume de contrôle}

On note le volume de contrôle $V$. On intègre l'équation de Poynting sur $V$, on note 
\begin{equation}
    U_{\mathrm{em}}(t)=\iiint_{V}u_{\mathrm{em}}(\vec{r},t)\d\tau
\end{equation}
l'énergie totale du champ régnant dans $V$,
\begin{equation}
    P_{S,\mathrm{rayon}}^{\mathrm{ext}}(t)=\iiint_{V}\div\vec{\Pi}\d\tau,
\end{equation}
et
\begin{equation}
    P_{\mathrm{champ}\to\mathrm{matiere}}(t)=\iiint_{V}P_{\mathrm{vol}}(\vec{r},t)\d\tau=\iiint_{V}\vec{j}\cdot\vec{E}\d\tau.
\end{equation}
Alors le bilan intégral d'énergie électromagnétique est
\begin{equation}
    \boxed{
        \frac{\d U_{\mathrm{em}}}{\d t}=-P_{S,\mathrm{rayon}}^{\mathrm{ext}}-P_{\mathrm{chp}\to\mathrm{mat}}.
    }
\end{equation}
En particulier, si $V$ est vide, on a $P_{\mathrm{chp}\to\mathrm{mat}}=0$ d'où 
\begin{equation}
    \boxed{
        \frac{\d U_{\mathrm{em}}}{\d t}=-P_{S,\mathrm{ray}}^{\mathrm{ext}}.
    }
\end{equation}
Si $\vec{\Pi}\cdot\vec{n}^{\mathrm{ext}}>0$, on a $P_{S,\mathrm{ray}}^{\mathrm{ext}}>0$ et donc $U_{\mathrm{em}}$ décroît avec le temps.

Dans un autre cas, si $P_{S,\mathrm{ray}}^{\mathrm{ext}}=0$ (comme dans une cavité électromagnétique, ou un four micro-ondes), alors 
\begin{equation}
    \boxed{
        \frac{\d U_{\mathrm{em}}}{\d t}=-P_{\mathrm{chp}\to\mathrm{mat}}.
    }
\end{equation}
Usuellement, on va avoir $P_{\mathrm{chp}\to\mathrm{mat}}>0$ et donc $U_{\mathrm{em}}$ décroît avec le temps. Cependant, dans un LASER, c'est l'inverse.

\subsection{Exemple: conducteur ohmique en régime stationnaire}

On se place dans le cas d'un conducteur ohmique cylindrique comme à la Figure~\ref{fig:puissance_dissipee_conducteur_ohmique}. L'expression du champ magnétique est
\begin{equation}
    \left\lbrace
        \begin{aligned}
            \vec{B}(r\geqslant a)&=\frac{\mu_0 i}{2\pi r}\vec{u}_{\theta},\\
            \vec{B}(r\leqslant a)&=\frac{\mu_0 i}{2\pi a}\left(\frac{r}{a}\right)\vec{u}_{\theta}.
        \end{aligned}
    \right.
\end{equation}
La loi d'ohm s'écrit
\begin{equation}
    \vec{E}=\frac{\vec{j}}{\sigma}=\frac{i}{\sigma\pi a^{2}}\vec{u}_z
\end{equation}
dans le conducteur (pour $r\leqslant a$). En régime permanent, on a $\frac{\d U_{\mathrm{em}}}{\d t}=0$ car les champs sont statiques. On a 
\begin{equation}
    \vec{\Pi}(r \leqslant a)=\vec{E}(r\leqslant a)\wedge \frac{\vec{B}(r\leqslant a)}{\mu_0}=\frac{\mu_0 i^{2}}{\mu_0\sigma 2\pi^{2}a^{3}}\left(\frac{r}{a}\right)(-\vec{u}_r).
\end{equation}
On a aussi
\begin{equation}
    \boxed{
        \vec{\Pi}(r=a)=-\frac{i^{2}}{\sigma 2\pi^{2}a^{3}}\vec{u}_r.
    }
\end{equation}
Le vecteur de Poynting est donc radial rentrant, et l'énergie pénètre latéralement dans le conducteur.
On a donc
\begin{equation}
    \boxed{
        P_{S,\mathrm{ray}}^{\mathrm{ext}}=\iint_{S}\vec{\Pi}(r=a)\cdot\vec{u}_r\,\d S=-\frac{i^{2}l}{\sigma\pi a^{2}}=-R i^{2}<0.
    }
\end{equation}
Enfin,
\begin{equation}
    P_{\mathrm{chp}\to\mathrm{mat}}=\iiint_{V}\vec{j}\cdot\vec{E}\d\tau=\sigma E^{2}\times\pi a^{2}l=\sigma\left(\frac{i}{\sigma\pi a^{2}}\right)^{2}\pi a^{2}l,
\end{equation}
donc
\begin{equation}
    \boxed{
        P_{\mathrm{chp}\to\mathrm{mat}}=\frac{l}{\sigma\pi a^{2}}i^{2}=Ri^{2}.
    }
\end{equation}
On a alors l'équilibre suivant:
\begin{equation}
    \boxed{
        P_{S,\mathrm{ray}}^{\mathrm{ext}}+P_{\mathrm{chp}\to\mathrm{mat}}=0.
    }
\end{equation}
Ainsi, toute la puissance apportée par le rayonnement est transférée à la matière.