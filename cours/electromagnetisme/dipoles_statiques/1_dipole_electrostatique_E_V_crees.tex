\section[Dipôle électrostatique: E et V créés]{Dipôle électrostatique : E et V créés}

\subsection{Le modèle du dipôle électrostatique}

On sait qu'il y a neutralité de la matière. Comment concilier cela avec les distributions de charges non nulles ?

\paragraph{Modèle d'une distribution neutre.} La distribution est constituée de charges positives $Q^{+}=q$ et négatives $Q^{-}=-q$. On se réfère à la Figure~\ref{fig:modele_distribution_neutre}. Alors pour avoir $Q=0$ (neutralité de la matière), on doit avoir $Q^{+}+Q^{-}=0$. Les barycentres des charges positives et négatives sont notées $G^{+}$ et $G^{-}$. Par définition, on définit le moment dipolaire par
\begin{equation*}
    \boxed{
        \vec{p}\coloneqq q\overrightarrow{G^{-}G^{+}},
    }
\end{equation*}
et son unité est le \si{\coulomb\metre}. Par exemple, pour l'atome d'hydrogène, $\vec{p}\parallel \vec{E}^{\text{ext}}$ et $\vec{p}\cdot\vec{E}^{\text{ext}}>0$. Pour une molécule présentant une symétrie centrale, on a $\vec{p}=\vec{0}$. Sinon, les molécules peuvent être polaires.

\paragraph{Ordre de grandeur.} On a $p\sim10^{-19}\times10^{-10}=10^{-29}\si{\coulomb\metre}$. On introduit donc le Debye $D$, où $1~D=\frac{1}{3}10^{-29}\si{\coulomb\metre}$. Pour un condensateur plan, on a $\left\lVert\vec{p}\right\rVert=Qe\sim10^{-13}\si{\coulomb\metre}$. Pour un potentiel de 10$\si{\volt}$, avec $e=1\si{\milli\metre}$ et $S=10\si{\centi\metre\squared}$, on a $Q=CU=\frac{\varepsilon_{0}SU}{e}\sim10^{-10}\si{\coulomb}$.

\begin{figure}
    \centering
    \tikzsetnextfilename{modele_distribution_neutre}
    \begin{tikzpicture}[scale=1]  
        \node at (-2,0) {$\times$};
        \node at (2,0) {$\times$};
        \node at (-2,0) [above] {$G^{-}$};
        \node at (2,0) [above] {$G^{+}$};
        \node at (-2,0) [below] {$-q$};
        \node at (2,0) [below] {$q$};
        \node[text=red] at (0,0){$\times$};
        \node[text=red] at (0,0)[below]{$O$};
        \draw[<->] (-1.75,0)--(1.75,0) node [above,midway] {$d$};

        \draw[<->,draw=blue] (2.5,0)--(3.5,0);
        \node[text=red] at (6,0) {$\times$};
        \draw[draw=red,text=red,->] (4,0)--(8,0) node [above,midway]{$\vec{p}$};
        \node[text=red] at (6,0) [below] {$O$};
    \end{tikzpicture}
    \caption{Modèle d'une distribution neutre.}    
    \label{fig:modele_distribution_neutre}
\end{figure}

\subsection{Symétries du problème}

Pour un dipôle, en coordonnées sphériques où il y a un potentiel $V(r,\theta,\varphi)$ et $\vec{E}(r,\theta,\varphi)$. Il y a cependant une invariance par symétrie de révolution par rapport à l'axe $(Oz)$, donc il n'y a pas de dépendance en $\varphi$. De plus, tout plan contenant $(Oz)$ est un plan de symétrie. Ainsi,
\begin{equation*}
    \boxed{
        \begin{aligned}
            &V(r,\theta),\\
            &\vec{E}(r,\theta)=\begin{pmatrix}
                E_r(r,\theta)\\ E_{\theta}(r,\theta)\\0
            \end{pmatrix}.
        \end{aligned}
    }
\end{equation*}

\subsection{Potentiel \texorpdfstring{$V$}{V} dans l'approximation dipolaire}

L'approximation dipolaire est $r\gg q$, où $r$ est la distance à laquelle se trouve le point $M$ dont on voit le dipôle. Dans ce cas, on a 
\begin{equation*}
    V(M)=\frac{q}{4\pi\varepsilon_{0}G^{+}M}-\frac{q}{4\pi\varepsilon_{0}G^{-}M}.
\end{equation*}
On écrit alors
\begin{align*}
    G^{+}M^{2}=\left\lVert \overrightarrow{G^{+}M}\right\rVert^{2}
    &=\left(\overrightarrow{G^{+}O}+\overrightarrow{OM}\right)^{2},\\
    &=\frac{d^{2}}{4}+r^{2}-2\underbrace{\overrightarrow{OG^{+}}\cdot\overrightarrow{OM}}_{\frac{d}{2}r\cos\theta},\\
    &=r^{2}\left[
        1-\frac{d\cos\theta}{r}+\frac{d^{2}}{4r^{2}}
    \right].
\end{align*}
Ainsi,
\begin{align*}
    \frac{1}{G^{+}M}
    &=(G^{+}M^{2})^{-1/2},\\
    &=\frac{1}{r}\left[1-\frac{d\cos\theta}{r}+\frac{d^{2}}{4r^{2}}\right]^{-1/2},\\
    &\approx\frac{1}{r}\left(1+\frac{d}{2r}\cos\theta\right).
\end{align*}
De même, on a 
\begin{equation*}
    \frac{1}{G^{-}M}\approx\frac{1}{r}\left(1-\frac{d}{2r}\cos\theta\right).
\end{equation*}
Finalement,
\begin{align*}
    V(r,\theta)
    &\approx\frac{q}{4\pi\varepsilon_0 r}\left(1+\frac{d}{2r}\cos\theta-1+\frac{d}{2r}\cos\theta\right),\\
    &=\frac{qd}{4\pi\varepsilon_0}\frac{\cos\theta}{r^{2}},\\
    &=\frac{p}{4\pi\varepsilon_0}\frac{\cos\theta}{r^{2}},\\
    &=\frac{\vec{p}\cdot\vec{r}}{4\pi\varepsilon_0 r^{3}}\propto\frac{1}{r^{2}}.
\end{align*}
Il  y a donc une décroissance en $\frac{1}{r^{2}}$, qui est beaucoup plus rapide que pour une charge qui est en $\frac{1}{r}$.

\subsection{Champ E dans l'approximation dipolaire}

En coordonnées sphériques, on a 
\begin{equation*}
    \vecgrad f(r,\theta,\varphi)=\begin{pmatrix}
        \frac{\partial f}{\partial r}\\\frac{1}{r}\frac{\partial f}{\partial \theta}\\\frac{1}{r\sin\theta}\frac{\partial f}{\partial\varphi}
    \end{pmatrix}.
\end{equation*}
Ainsi, on trouve
\begin{equation*}
    \boxed{
        \vec{E}=-\vecgrad V(r,\theta)=\frac{p}{4\pi\varepsilon_0 r^{3}}\begin{pmatrix}
            2\cos\theta\\\sin\theta\\0
        \end{pmatrix}
    }\propto\frac{1}{r^{3}}.
\end{equation*}

\subsection{Allure des lignes de champ et des surfaces équipotentielles dans l'approximation dipolaire}

Soit un déplacement $\vec{\rmd l}$ le long d'une ligne de champ $\vec{\rmd l}\parallel\vec{E}$ \emph{i.e.~}$\vec{\rmd l}\wedge\vec{E}=\vec{0}$. On trouve alors $-0-r\sin\theta E_{\theta}\rmd\varphi=0$, d'où $\rmd\varphi=0$ donc $\varphi=$constante le long d'une ligne de champ. De plus, on a 
\begin{equation*}
    -\rmd rE_{\theta}-r\rmd\theta E_r=0,
\end{equation*}
d'où
\begin{equation*}
    \rmd r\sin\theta=2r\cos\theta\rmd\theta.
\end{equation*}
On trouve alors $\frac{\rmd r}{\rmd\theta}=2r\frac{\cos\theta}{\sin\theta}$, d'où $r(\theta)=r_0\sin^{2}\theta$. Alors, les équipotentielles dans l'approximation dipolaire ont pour équation $V=\frac{p\cos\theta}{4\pi\varepsilon_0 r^{2}}=V_0$.