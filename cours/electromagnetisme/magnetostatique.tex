\chapter{Magnétostatique}

Les courants sont stationnaires : $\vec{j}(\vec{r},t)=\vec{j}(\vec{r})$. Ils circulent dans des conducteurs neutres : $\rho(\vec{r})=0$. Ainsi $\vec{B}=\vec{B}(\vec{r})$.

\minitoc

% Première section : Flux conservatif du champ magnétique : formulations intégrale et locale
\subimport{magnetostatique/}{1_flux_conservatif_champ_magnetique_formulations_integrale_locale.tex}

% Deuxième section : Circulation du champ magnétique : théorème d'Ampère intégral et local
\subimport{magnetostatique/}{2_circulation_champ_magnetique_theoreme_ampere_integral_local.tex}

% Troisième section : Topographie du champ magnétique
\subimport{magnetostatique/}{3_topographie_champ_magnetique.tex}