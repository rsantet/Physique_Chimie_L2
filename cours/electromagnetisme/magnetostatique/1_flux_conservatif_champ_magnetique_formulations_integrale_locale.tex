\section[Flux du champ magnétique]{Flux conservatif du champ magnétique : formulations intégrale et locale}
    \subsection{Nullité du flux magnétique à travers une surface fermée}

        On a le postulat suivant :
        \begin{equation*}
            \boxed{
                \oiint_{S}\vec{B}\cdot\vec{n}^{\text{ext}}\rmd S=0.
            }
        \end{equation*}

        Ainsi, il n'existe pas de \og masses\fg~ni de \og charges\fg~magnétiques.

    \subsection{Flux de \texorpdfstring{$\vec{B}$}{B} à travers une section d'une tube de champ}

        On se réfère à la Figure~\ref{fig:flux_B_section_tube_champ}. On note $\varphi_{i}=\iint_{S_i}\vec{B}\cdot\vec{n}_i\rmd S_i$. Alors
        \begin{equation*}
            \oiint_{S_1\cup S_2\cup\Sigma}\vec{B}\cdot\vec{n}^{\text{ext}}\rmd S=0,
        \end{equation*}
        implique
        \begin{equation*}
            \boxed{
                \varphi_1=\varphi_2.
            }
        \end{equation*}

        \begin{figure}
            \centering
            \tikzsetnextfilename{flux_B_section_tube_champ}
            \begin{tikzpicture}[scale=1]  
                \draw[draw=red,text=red] (-2,1)--(2,1) node [above, pos=0.8]{$\Sigma$};
                \draw[red] (-2,-1)--(2,-1);
                \draw[red,dashed] (-4,1.5) to[bend right=10] (-2,1);
                \draw[red,dashed] (-4,-1.5) to[bend left=10] (-2,-1);
                \draw[red,dashed] (4,1.5) to[bend left=10] (2,1);
                \draw[red,dashed] (4,-1.5) to[bend right=10] (2,-1);
                \draw[red,pattern=north east lines, pattern color=red] (-2,0) ellipse (0.2 and 1);
                \draw[red,pattern=north east lines, pattern color=red] (2,0) ellipse (0.2 and 1);
                \draw[red, dashed] (-4,0)--(-2,0);
                \draw[red, dashed] (2,0)--(4,0);
                \draw[red] (-2,0)--(2,0);
                \node[text=red] at (-2.5,-0.5){$S_1$};
                \node[text=red] at (2.5,-0.5){$S_2$};
                \draw[-latex,red,text=red] (-2,0.4)--++(0.5,0) node [above] {$\vec{n}_{1}$};
                \draw[-latex,red,text=red] (2,0.4)--++(0.5,0) node [above] {$\vec{n}_{2}$};
            \end{tikzpicture}
            \caption{Flux du champ magnétique à travers une section d'un tube de champ.}
            \label{fig:flux_B_section_tube_champ}
        \end{figure}

    \subsection{Resserrement des lignes de champ magnétiques}
        
        En notant $\bar{B}_i$ la valeur moyenne sur $S_i$ et en supposant $S_1>S_2$, on a
        \begin{equation*}
            \frac{\bar{B}_2}{\bar{B}_1}=\frac{S_1}{S_2}>1.
        \end{equation*}
        Ainsi, $\left\lVert\vec{B}\right\rVert$ est plus intense où les lignes de champ sont plus serrées.

    \subsection{Équation locale de Maxwell-Thompson}

        D'après le théorème d'Ostrogradski, on a 
        \begin{equation*}
            \boxed{
                \oiint_{S}\vec{B}\cdot\vec{n}^{\text{ext}}\rmd S=0\Longleftrightarrow\mathrm{div}\vec{B}=0.
            }
        \end{equation*}

        L'équation est locale et universelle.