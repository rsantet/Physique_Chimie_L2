\section{Topographie du champ magnétique}
    \subsection{Fermeture et orientation des lignes de contour}

        Comme le flux du champ est nul, toute ligne de champ magnétique est fermée sur elle-même. De plus, comme $\vec{\mathrm{rot}}\vec{B}=\mu_{0}\vec{j}$, les lignes de champ tournent autour des courants dans le sens donné par la règle du tire-bouchon.

    \subsection{Comparaison des lignes de champ électriques et magnétiques}
        
        Une ligne de champ fermée sur elle-même est une ligne de champ magnétique, tandis qu'une ligne de champ convergente ou divergente en un point est une ligne de champ électrique. Au voisinage des sources, les lignes de champ électriques divergent ou convergent radialement, tandis que les lignes de champ magnétiques tournent autour de la source. Loin des sources, les deux champs sont régis par les mêmes équations ($\vec{\mathrm{rot}}\vec{E}=\vec{\mathrm{rot}}\vec{B}=\vec{0}$ et $\mathrm{div}\vec{B}=\mathrm{div}\vec{E}=0$), la distinction n'est donc pas évidente a priori.