\chapter[Dispersion et atténuation]{Dispersion et atténuation : des plasmas aux métaux}

Le plasma est le quatrième état de la matière : c'est un gaz partiellement ou totalement ionisé. On en retrouve de manière naturelle dans la ionosphère ou dans la matière stellaire, et de manière artificielle dans les tokamaks (réacteurs de fusion nucléaire), laser mégajoules, plasmas basse témpérature (circuits électriques). L'étude des plasmas est la magnéto-hydro-dynamique.

On retrouve dans les plasmas des cations ($+\rme,M,n^{+}$), où $n^{+}$ est ee nombre de particules par unité de volume (en $\si{\per\metre\cubed}$), et des électrons ($-\rme,m,n^{-}$). Dans la ionosphère (entre 60 et 800$\si{\kilo\metre}$ d'altitude), on observe~$n^{-}=n^{+}=10^{10}$ à $10^{12}\si{\per\metre\cubed}$.

Les mtéaux sont formés d'un réseau cristallin de cations (par exemple l'ion cuivre \ce{Cu^+}) baignant dans une mer (gel ou \og jelly\fg~en anglais) d'électrons libres (on dit aussi de conduction). Dans ce cas,~$n^{-}$ représente le nombre d'électrons libres par unité de volume, et~$n^{+}$ le nombre de cations par unité de volume. Dans le cuivre, on observe~$n^{+}=n^{-}\sim 10^{29}\si{\per\metre\cubed}$.


\minitoc

% Première section : Modèle de la conductivité complexe : plasmas et métaux
\subimport{dispersion_attenuation_plasmas_metaux/}{1_modele_conductivite_complexe_pasmas_metaux.tex}