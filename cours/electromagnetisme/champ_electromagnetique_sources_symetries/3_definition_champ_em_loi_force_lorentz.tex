\section[La loi de force de Lorentz]{Définition du champ électromagnétique : la loi de force de Lorentz}

    On se demande quelle est l'action (à distance) de la distribution de charges et courants sur une particule chargée $q$ de vitesse $\vec{v}$, voir la Figure~\ref{fig:definition_champ_em_loi_force_lorentz}.

    \begin{figure}
        \centering
        \tikzsetnextfilename{definition_champ_em_loi_force_lorentz}
        \begin{tikzpicture}[scale=1]  
            \draw (-3,1) ellipse (1.5 and 2.5);
            \draw [pattern=north east lines, pattern color=blue] (-2.5,0.5) circle (0.5);
            \draw [latex-,text=blue] (-2.5,0.5) -- (1,-0.5) node [right] {$\rho(\vec{r},t),\vec{j}(\vec{r},t)$};
            \draw [-latex, text=red, draw=red] (1,2) --++(0.75,-0.75) node [right] {$\vec{v}$};
            \node[red] at (1,2) {$\bullet$};
            \node[red] at (1,2) [above right] {$q$};
            \node[text width=3cm] at (-3,-2.5) {$\mathcal{D}$: distribution de charges et/ou de courants};
        \end{tikzpicture}
        \caption{Définition du champ électromagnétique : la loi de force de Lorentz.}    
        \label{fig:definition_champ_em_loi_force_lorentz}
    \end{figure}

    Cette action se fait via le champ électromagnétique [$\vec{E},\vec{B}$], conséquence directe de [$\rho,\vec{j}$]. La loi de force de Lorentz s'écrit
    \begin{equation}
        \boxed{
            \vec{F}_L = q\left(\vec{E}+\vec{v}\wedge\vec{B}\right).
        }
    \end{equation}
    C'est un postulat. $\vec{E}$ est polaire et est considéré comme un \og vrai\fg~vecteur. $\vec{B}$ est axial est est considéré comme un \og pseudo-vecteur\fg.

    \paragraph{Force de Lorentz méso/macroscopique : distribution continue de charges et courants.}

        Il y a des porteurs de charges de type $k$ : charges~$q_k$, nombre par unité de volume $n_k [\si{\metre\per\cubed}]$ et vitesse moyenne $\langle\vec{v_k}\rangle$. Soit $\d V$ un volume mésoscopique contenant ces porteurs. Quelle est la force $\d\vec{F}_{em}$ subie par ce volume ? Un charge $k$ subit en moyenne 
        \begin{equation}
            \vec{F}_{L_k}=q_k\left(\vec{E}+\langle\vec{v_k}\rangle\wedge \vec{B}\right).
        \end{equation}

        Dans $\d V$, il y a $n_k\d V$ porteurs $k$. Ainsi, ils subissent
        \begin{equation}
            n_k q_k\left(\vec{E}+\langle \vec{v_k}\rangle\wedge \vec{B}\right)\d V.
        \end{equation}
        En sommant sur tout les porteurs $k$, on a 
        \begin{align}
            \d\vec{F}_{\text{em}}
            &=
            \left[
                \sum_{\neq~k}n_kq_k\left(\vec{E}+\langle\vec{v_k}\rangle\wedge\vec{B}\right)
            \right]\d V,\\
            &=
            \left(
                \left(\sum_{\neq~k}n_k q_k\right)\vec{E}+\left(\sum_{\neq~k}n_kq_k\langle\vec{v_k}\rangle\right)\wedge\vec{B}
            \right),\\
            &=
            \left(\rho\vec{E}+\vec{j}\wedge\vec{B}\right)\d V.
        \end{align}

        Ainsi, la force volumique est 
        \begin{equation}
            \boxed{
                \vec{f_{\text{vol}}^{\text{em}}}=\rho\vec{E}+\vec{j}\wedge\vec{B}.
            }
        \end{equation}

        \begin{example}
            Dans un métal, c'est le force de Laplace : on a $\rho=0$ d'où 
            \begin{equation}
                \boxed{
                    \d\vec{F}_{\text{em}}=(\vec{j}\wedge\vec{B})\d V.
                }
            \end{equation}

            Pour une géométrie filiforme, on a $\vec{j}\d V=js\d l=i\vec{\d l}$, où $s$ est la section du fil. Ainsi
            \begin{equation}
                \boxed{
                    \d\vec{F_{\text{em}}}=i\vec{\d l}\wedge\vec{B}.
                }
            \end{equation}
        \end{example}