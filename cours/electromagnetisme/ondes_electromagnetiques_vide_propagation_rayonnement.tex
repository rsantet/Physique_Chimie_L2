\chapter[Ondes dans le vide]{Ondes électromagnétiques dans le vide. Propagation et rayonnement}

On a vu que, du point de vue du vide, sans charges ni courants, le champ électromagnétique est régi par
\begin{equation*}
    \boxed{
        \begin{aligned}
            \vec{\Delta E}(\vec{r},t)-\frac{1}{c^{2}}\frac{\partial ^{2}\vec{E}(\vec{r},t)}{\partial t^{2}}&=\vec{0},\\
            \vec{\Delta B}(\vec{r},t)-\frac{1}{c^{2}}\frac{\partial ^{2}\vec{B}(\vec{r},t)}{\partial t^{2}}&=\vec{0},
        \end{aligned}
    }
\end{equation*}
où $c\coloneqq 299\,792\,458\,\si{\metre\per\second}$.

L'équation de d'Alembert en trois dimensions est
\begin{enumerate}[label=(\roman*)]
    \item linéaire;
    \item invariante par renversement du temps car seule une dérivée seconde du temps intervient;
    \item l'espace et le temps sont couplés via la constante $c$.
\end{enumerate}


\minitoc

% Première section : Solutions de l'équation de d'Alembert en OPP
\subimport{ondes_electromagnetiques_vide_propagation_rayonnement/}{1_solution_equation_d_alembert_opp.tex}

% Deuxième section : La solution en OPPH
\subimport{ondes_electromagnetiques_vide_propagation_rayonnement/}{2_solution_en_opph.tex}

% Troisième section : Propagation de l'énergie par une OPP(H)
\subimport{ondes_electromagnetiques_vide_propagation_rayonnement/}{3_propagation_energie_opph.tex}

% Quatrième section : Rayonnement dipolaire électrique
\subimport{ondes_electromagnetiques_vide_propagation_rayonnement/}{4_rayonnement_dipolaire_electrique.tex}